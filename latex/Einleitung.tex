% !TEX root = Facharbeit_Mathe_Grundlagen_ML-DL.tex
\chapter{Einführung}
In meiner Facharbeit möchte ich mich mit dem Thema „Künstliche Intelligenz“ intensiv auseinandersetzen. Die USA, China und Europa kämpfen um die Vorherrschaft, in diesem neuen Gebiet der Informatik, welche die Zukunft in unserem Leben revolutionieren wird. Doch was bedeutet eigentlich künstliche Intelligenz? Was macht dieses Thema so interessant und Zukunftsrelevant, dass nun sogar Deutschland drei Milliarden Euro investieren möchte? Erstmal ist die Künstliche Intelligenz (KI) ein Teilgebiet der Informatik, die ihren Ursprung im 20. Jahrhundert hatte. Dieses Teilgebiet befasst sich mit maschinellem Lernen und der Automatisierung intelligenten Verhaltens. Schon heute findet man vielseitige Anwendungen von KI. Beispiele dafür sind die Gesichtserkennung, die Google-Sucherkennung und Werbung. Zahlreiche Anwendungsbereiche in Naturwissenschaft, Medizin, Technik, Informatik und im Medienbereich sind möglich. Das spannende an KI, Machine Learning und Deep Learning ist, dass es auf Grundlagen der Mathematik basiert. Das heißt, dass es auch einem Schüler möglich ist, mit dem mathematischen Wissen für die Sekundarstufe II, die Grundzüge der KI nachzuvollziehen. Das motiviert zum einen worin die Mathematik zum Teil ihre praktische Anwendung findet und was man als Schüler leisten kann. 
